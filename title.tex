\def\hrf#1{\hbox to#1{\hrulefill}} %команда для задания линеек
\newcommand{\sfs}{\fontsize{14pt}{15pt}\selectfont}
\sfs % размер шрифта и расстояния между строками
\thispagestyle{empty}

\vspace{10mm}
\begin{flushright}
	\Large На правах рукописи 
%  \textit{моя подпись}
\end{flushright}

\vspace{30mm}
\begin{center}
{\Large\bf Бондарева Ольга Васильевна}
\end{center}

\vspace{30mm}
\begin{center}
{\bf \LARGE Молекулярные адаптации грызунов к подземному образу жизни на примере подсемейства полевочьих (Arvicolinae, Rodentia)
\par}
%{\bf \LARGE \textit{ Macoma~balthica}~(Linnaeus,~1758) в осушной зоне Белого и Баренцева морей: рост, организация и  долгосрочная динамика поселений
%\par}

\vspace{30mm}
{\Large
Специальность 03.02.04 --- <<Зоология>>
}

\vspace{15mm}
\LARGE Автореферат\par
\Large диссертации на соискание учёной степени\par
кандидата биологических наук
\end{center}

\vspace{40mm}
\begin{center}
{\Large Санкт-Петербург --- 2022}
\end{center}

\newpage
% оборотная сторона обложки
\thispagestyle{empty}
\noindent Работа выполнена в Федеральном государственном бюджетном учреждении науки <<Зоологический институт Российской академии наук>>

\begin{table} [h]  
  \begin{tabular}{ll}   
   \makecell[l]{\sfs Научный руководитель:\\~} &
   \makecell*[{{p{12cm}}}]{\textbf{\sfs Абрамсон Наталья Иосифовна} \\ \sfs
   кандидат биологических наук}
      
\vspace{1mm} \\

   \makecell[l]{\sfs Официальные оппоненты: \vspace{6.65cm}} &
   \makecell[{{p{12cm}}}]{   
   \sfs \textbf{Чабовский Андрей Всеволодович,} \\
   \sfs доктор биологических наук, \\
   \sfs Институт проблем экологии и эволюции им. А.Н. Северцова РАН,  \\
   \sfs зав. лаборатории популяционной экологии. \vspace{5mm} \\
   \sfs \textbf{Недолужко Артем Валерьевич,} \\
   \sfs кандидат биологических наук, \\
   \sfs Европейский университет в Санкт-Петербурге, \\
   \sfs зав. лаборатории палеогеномики
   }

\vspace{1mm} \\

   \makecell[l]{\sfs Ведущая организация:\\~\\~\\~} &
   \makecell*[{{p{11cm}}}]{\sfs
   \textbf{Научно-исследовательский институт биологии Иркутского государственного университета}
   }
  \end{tabular}  
\end{table}

%\noindent Защита состоится <<24>> марта 2016~г.~в~13 часов на~заседании диссертационного совета  Д.501.001.55 при  Московском государственном университете имени~М.\:В.~Ломоносова по адресу: 119899, Москва, Ленинские горы, дом~1, МГУ, корп.~12, Биологический факультет,  ауд.~389. Тел.~+7(495)939-25-73, эл.почта: dissovet\_00155@mail.ru


\vspace{5mm}
%\noindent С  диссертацией  можно  ознакомиться  в  библиотеке  Биологического  факультета Московского  государственного  университета  имени~М.\:В.~Ломоносова и  на  сайте http://www.bio.msu.ru/

\vspace{5mm}
\noindent Автореферат разослан <<\hrf{2em}>> \hrf{6em} 20\hrf{2em}.

\vspace{5mm}
%\begin{table} [h]
%  \begin{tabular}{p{8cm}cr}
%    \begin{tabular}{p{8cm}}
%      \sfs Ученый секретарь  \\
%      \sfs диссертационного совета  \\
%%      \sfs  Д.501.001.55,  \\
%%      \sfs кандидат биологических наук
%    \end{tabular} 
%    & \begin{tabular}{c}
%       %\includegraphics [height=2cm] {Kartasheva_sign.jpg} 
%    \end{tabular} 
%    & \begin{tabular}{r}
%       \\
%       \\
%%       \sfs Карташева Наталия Васильевна
%    \end{tabular} 
%  \end{tabular}
%\end{table}
\newpage
