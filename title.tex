\def\hrf#1{\hbox to#1{\hrulefill}} %команда для задания линеек
\newcommand{\sfs}{\fontsize{14pt}{15pt}\selectfont}
\sfs % размер шрифта и расстояния между строками
\thispagestyle{empty}

\vspace{10mm}
\begin{flushright}
	\Large На правах рукописи 
%  \textit{моя подпись}
\end{flushright}

\vspace{30mm}
\begin{center}
{\Large\bf Бондарева Ольга Васильевна}
\end{center}

\vspace{30mm}
\begin{center}
{\bf \LARGE Молекулярные адаптации грызунов к подземному образу жизни на примере подсемейства полевочьих (Arvicolinae, Rodentia)
\par}
%{\bf \LARGE \textit{ Macoma~balthica}~(Linnaeus,~1758) в осушной зоне Белого и Баренцева морей: рост, организация и  долгосрочная динамика поселений
%\par}

\vspace{30mm}
{\Large
Специальность 1.5.12. --- <<Зоология>>
}

\vspace{15mm}
\LARGE Автореферат\par
\Large диссертации на соискание учёной степени\par
кандидата биологических наук
\end{center}

\vspace{40mm}
\begin{center}
{\Large Санкт-Петербург --- 2022}
\end{center}

\newpage
% оборотная сторона обложки
\thispagestyle{empty}
\noindent Работа выполнена в Федеральном государственном бюджетном учреждении науки Зоологический институт Российской академии наук

\begin{table} [h]  
  \begin{tabular}[t]{ll}   
   \makecell[l]{\sfs Научный руководитель:\\~} &
   \makecell*[{{p{12cm}}}]{\sfs кандидат биологических наук, \\ 
   \sfs \textbf{Абрамсон Наталья Иосифовна}}
      
\vspace{4mm} \\

   \makecell[l]{\sfs Официальные оппоненты: \\~ \\~ \\~ \\~ \\~ \\~ \vspace{6mm} \\~ \\~ \\~ \\~ \\~} &
   \makecell[{{p{12cm}}}]{   
   \sfs \textbf{Чабовский Андрей Всеволодович,} \\
   \sfs доктор биологических наук, \\
   \sfs Федеральное государственное бюджетное учреждение науки Институт проблем экологии и эволюции им. А.Н. Северцова РАН,  \\
   \sfs заведующий лабораторией популяционной экологии. \vspace{5mm} \\
   \sfs \textbf{Недолужко Артем Валерьевич,} \\
   \sfs кандидат биологических наук, \\
   \sfs Автономная некоммерческая образовательная организация высшего образования Европейский университет в Санкт-Петербурге, \\
   \sfs лаборатория палеогеномики, директор по развитию
   }

\vspace{4mm} \\

   \makecell[l]{\sfs Ведущая организация:\\~\vspace{2mm}\\~} &
   \makecell*[{{p{11cm}}}]{\sfs
   \textbf{Федеральное государственное бюджетное образовательное учреждение высшего образования Иркутский государственный университет}
   }
  \end{tabular}  
\end{table}

\noindent Защита состоится <<  >>     2023~г.~в~   часов на~заседании диссертационного совета 24.1.026.02 при Федеральном государственном
бюджетном учреждении науки Зоологический институт Российской академии наук по адресу: 199034, г. Санкт-Петербург, Университетская наб., д. 1. 


\vspace{5mm}
\noindent С диссертацией можно ознакомиться в библиотеке и на сайте Зоологического института Российской академии наук: https://www.zin.ru/boards/24.1.026.02/theses.html

\vspace{5mm}
\noindent Автореферат разослан <<\hrf{2em}>> \hrf{6em} 20\hrf{2em}.

\vspace{5mm}
%\begin{table} [h]
%  \begin{tabular}{p{8cm}cr}
%    \begin{tabular}{p{12cm}}
%      \sfs Ученый секретарь  \\
%      \sfs диссертационного совета  \\
%      \sfs кандидат биологических наук
%    \end{tabular} 
%%    & \begin{tabular}{c}
%%       %\includegraphics [height=2cm] {Kartasheva_sign.jpg} 
%%    \end{tabular} 
%    & \begin{tabular}{r}
%       \\
%       \\
%       \sfs Петрова Екатерина Анатольевна
%    \end{tabular} 
%  \end{tabular}
%\end{table}
\begin{table}[h]
\begin{tabular}{b{.6\linewidth}b{.4\linewidth}}
\large \mbox{Ученый секретарь} \linebreak
\mbox{диссертационного совета} \linebreak
\mbox{кандидат биологических наук}
&  \large Петрова Екатерина Анатольевна \\
\end{tabular}
\end{table}


\newpage
