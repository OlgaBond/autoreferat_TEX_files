%\section*{Общая характеристика работы}
%\fontsize{14pt}{15pt}\selectfont
% paragraph - меньше шрифт
% зеленые гиперссылки

\hspace{-10mm}\textbf{\Large Общая характеристика работы}

\vspace{3mm}
\textbf{Актуальность темы исследования.} К подземным грызунам относят примерно 250 видов, проводящих всю или почти всю свою жизнь под землей. Они распространены по всем континентам за исключением Австралии и Антарктиды (\cite{Fang2015}). Уход под землю помогает избежать открытых контактов с хищниками и сильных температурных колебаний, но приводит к возникновению новых стрессовых факторов: темнота, кислородная недостаточность и гиперкапния, повышенный инфекционный фон. Существование в этих условиях приводит к формированию сходных морфо-физиологических адаптаций у филогенетически далеких форм. Молекулярные основы этого процесса остаются не до конца понятными. %К настоящему моменту в открытых базах данных уже накопилось достаточное количество последовательностей как отдельных генов, так и полногеномных данных, позволяющих проводить сравнения и поиск следов адаптаций на различных таксономических уровнях.

Подсемейство полевочьи (Arvicolinae) -- одна из самых молодых и многочисленных групп отряда, распространенная практически во всех ландшафтных зонах Северного полушария. Они представляет собой удобную модель для тестирования гипотез о темпах и формах адаптивной эволюции, поскольку за относительно недолгий период эволюции группы ее представители независимо переходили к жизни под землей. Предыдущие исследования молекулярных адаптаций к подземному образу жизни выполнялись на немногочисленных и полностью подземных представителях филогенетически далеких таксонов из разных семейств и подотрядов (слепыши, землекопы, туко-туко). Полевки, в свою очередь, предоставляют уникальные возможности для изучения этих механизмов за счет сравненения близкородственных пар подземных и наземных видов в пределах одного семейства. 


% морфо-фзиологическе адаптации изучались. Значительно меньше мы знаем о молекулярных механизмах, которые изучались сильно хуже. 

%Подземные грызуны являются прекрасным модельным объектом эволюционной биологии для изучения адаптаций к подземному образу жизни. Более того, сравнение их с родственными наземными видами с использованием молекулярных подходов может способствовать выявлению процессов формирования адаптаций, начиная с молекулярного уровня. 

%рассказ о подсемействе, его преимущества (морфологически изучалось, молекулярно очень плохо, можем изучить начальные стадии адаптациии есть градации адаптации к подземному образу жизни) и том, как изучалось на других видах и в чем были проблемы


\paragraph{Степень разработанности темы исследования.} Несмотря на очень интенсивную историю исследования подсемейства, мало что известно относительно молекулярных механизмов их быстрой адаптивной радиации. Среди подземных полевочьих изучены только молекулярные адаптации \textit{Lasiopodomys mandarinus} Milne-Edwards, 1871. Однако, авторы исследований проводили сравнения не в рамках подсемейства, а с отдельными подземными грызунами из других филогенетически очень далеких семейств: \textit{Heterocephalus glaber} Rüppell, 1842, \textit{Fukomys damarensis} Ogilby, 1838 и несколькими видами рода \textit{Spalax} Guldenstaedt, 1770 (\cite{Sun2020}; \cite{Sun2018a}; \cite{Dong2020}). Несмотря на то, что в подсемействе есть другие виды, которые независимо перешли к подземному образу жизни, исследования их молекулярных адаптаций не проводилось. Также не делались сравнения и анализ конвергенции молекулярных признаков с другими более эволюционно древними подземными грызунами.

%Предыдущие исследования молекулярных адаптаций к подземному образу жизни выполнялись на немногочисленных и полностью подземных представителях филогенетически далеких таксонов из разных семейств и подотрядов (слепыши, землекопы, туко-туко).

\paragraph{Цели и задачи работы.} Целью данной работы является проведение молекулярно-генетических сравнениий филогенетически независимых подземных форм подсемейства полевочьих (Arvicolinae, Cricetidae, Rodentia) и их наземных сестринских таксонов и выявление следов отбора при освоении подземной ниши на молекулярном уровне.
%\vspace{0pt plus0.5fill}

\vspace{3mm}
Для достижения поставленной цели были сформированы следующие \textbf{задачи}:

\begin{enumerate}
	\item Сравнить направление и силу отбора для гена \textit{CYTB}, белок-кодирующих митохондриальных и части ядерных генов у подземных и наземных грызунов;
	\item Провести поиск параллельных аминокислотных замен в ядерных и митохондриальных генах в независимых линиях подземных полевочьих;
	\item Выявить функции генов с отличиями в силе и направлении отбора относительно наземных грызунов и параллельными аминокислотными заменами, определить биохимические процессы, в которые они вовлечены;
	\item Сравнить количество генов со следами адаптации к подземному образу жизни среди подземных представителей подсемейства Arvicolinae;
	\item Провести сравнение геномных изменений у эволюционно молодых подземных полевочьих с представителями эволюционно более древних семейств подземных грызунов. 
\end{enumerate}

\paragraph{Научная новизна.} В рамках работы впервые проведены масштабные исследования представителей подсемества полевочьи (Arvicolinae, Rodentia) для поиска следов конвергентной эволюции в филогенетически независимых линиях подземных грызунов. Исследование выполнено на нескольких уровнях: от анализа отдельного филогенетического маркера \textit{CYTB} до пула ядерных белок-кодирующих генов. Впервые проанализированы паттерны аминокислотных замен и выявлены сайты с параллельными заменами, характерными для подземных грызунов. Для всех белок-кодирующих митохондриальных и ряда ядерных (112) генов проведена оценка силы и направления отбора несколькими методами (codeml branch model, RELAX, aBSREL). В ходе выполнения работы в лаборатории эволюционной геномики и палеогеномики ЗИН РАН получено 36 новых митохондриальных геномов и более 15 транскриптомов, что представляет существенный вклад для дальнейшего изучения эволюционной истории подсемейства. 

\paragraph{Теоретическая и практическая значимость работы.} В работе получены фундаментальные данные, описывающие молекулярные адаптации к подземному образу жизни представителей подсемейства полевочьи. Также впервые дана сравнительная характеристика различий в уровне отбора между видами, независимо перешедшими к  подземному образу жизни. Обнаруженные гены с измененным уровнем отбора и параллельными заменами могут служить источником для более детального изучения адаптивной и эволюционной физиологии. Собранные и опубликованные в открытом доступе митохондриальные геномы и транскриптомы будут использованы в работах по филогеографии, филогении и изучении других эволюционных процессов внутри подсемейства Arvicolinae сотрудниками как Зоологического института РАН, так и учебных и научных заведений всего мира. Результаты исследования могут быть использованы в курсах лекций по эволюционной биологии в вузах, школах и секциях дополнительного образования.

\paragraph{Положения, выносимые на защиту.}
\begin{enumerate}
	\item Наблюдается ослабление уровня отбора в большинстве митохондриальных белок-кодирующих генов у подземных форм полевочьих. 
	\item У подземных форм подсемейства полевочьи присутствуют параллельные аминокислотные замены в генах, которые вовлечены в процессы адаптации к низкой концентрации кислорода.
	\item Интенсивность уровня отбора на митохондриальные и ядерные гены коррелирует со степенью специализации к подземному образу жизни и не зависит от возраста таксона.
	\item Направления адаптивной изменчивости у подземных форм полевочьих имеют тот же характер, что и в других древних специализированных семейств подземных грызунов Spalacidae, Ctenomyidae, Bathyergidae.

\end{enumerate}

\paragraph{Личный вклад автора.} Личный вклад автора работы состоит в сборе материала на территории Даурского заповедника (2018 г.) и Якутии (2019 г.), обработке материала, проведении всех анализов по изучению уровня и направления отбора. Подготовка публикации осуществлялась автором как самостоятельно, так и в соавторстве с коллегами. В большинстве публикаций автор является первым автором, где ему принадлежит ведущая роль как при проведении исследований, так и при подготовке рукописей (не менее 80\%).

\paragraph{Степень достоверности и апробации результатов.} Материалы диссертации представлены на 7 международных конференциях и конгрессах: международных конференциях по вычислительной биологии MCCMB в 2017 и 2019 г. (Москва, Россия); XI международной конференции по биоинформатике и системной биологии BGRS (Новосибирск, 2018); 16 международной конференции Rodens et Spatium (Потсдам, Германия, 2018); VII международном конгрессе общества генетиков и селекционеров ВОГиС (Санкт-Петербург, 2019); 45 конгрессе федерации европейского биохимического сообщества (FEBS) (дистанционно, 2021); XI Съезде Териологического общества при РАН (2022). Материалы также представлены на итоговой отчетной сессии ЗИН РАН в 2020 году, семинарах лаборатории эволюционной геномики факультета биоинформатики и биоинженерии МГУ (Москва, 2018, 2021) и биоинформатическом семинаре Университета ИТМО (2022).

\paragraph{Публикации.} По теме диссертации опубликовано 14 работ. Из них 7 статей в журналах из списка, рекомендованного ВАК, в том числе 7 на английском языке в журналах, индексируемых международными базами данных Scopus и Web of Science Core Collection; 7 тезисов.

\paragraph{Структура и объем диссертации.} Работа состоит из введения, трех глав, заключения, выводов и списка литературы. Основная часть работы изложена на 73 страницах, содержит 16 рисуноков и 13 таблиц. Список литературы включает 148 наименований, из которых 7 на русском языке и 141 -- на английском. Приложения к работе содержит 4 таблицы на 16 страницах.

%\underline{\textbf{Благодарности}}
\paragraph{Благодарности.}
%\begin{small}
В первую очередь хочу поблагодарить мою научную руководительницу Наталью Иосифовну Абрамсон за помощь и поддержку при выполнении диссертации, ценные советы и доверие в выборе методик анализа. Отдельную благодарность хотелось бы выразить всему коллективу лаборатории эволюционной геномики и палеогеномики ЗИН РАН за неоценимый вклад в мое зоологическое образование, освоение филогенетических и филогеографических методик: Семену Бодрову, Татьяне Петровой и Евгению Генельт-Яновскому. За помощь в изучении биоинформатических подходов искренне благодарю Институт биоинформатики, а за обсуждение полученных результатов и ценные замечания -- А.В. Сморкачеву и сотрудников ИППИ РАН, ФББ МГУ и ИОГеН РАН: Надежду Потапову, Артема Касьянова, Алексея Пенина, Марию Логачеву, Егора Базыкина и А.С. Кондрашова.   

За моральную поддержку во время написания работы хочу поблагодарить в первую очередь своего супруга Станислава Бондарева, а также Ольгу Бочкареву, Александру Пантелееву, Анну Гнетневу, Анну Ганюкову, Евгения Генельт-Яновского, членов моей семьи, друзей и коллектив Института биоинформатики. Отдельную благодарность хочу выразить моей бабушке, Паненковой Галине Ильиничне, которая до конца верила, что у меня все получится. 

За помощь в организации рабочего времени выражаю огромную благодарность Татьяне Смирновой и Екатерине Копейкиной, без которых получение результатов и написание диссертации заняло бы гораздо больше времени. Отдельную благодарность выражаю композиторам компаний CD Projekt Red, Guerrilla Games и FromSoftware, под произведения которых данная работа была написана. 

Работа выполнена в рамках темы государственного задания №~AAAA–A19–119020790106–0 в лаборатории эволюционной геномики и палеогеномики ЗИН РАН. Исследования поддержаны грантами РФФИ №18-04-00730, №15-04-04602, №18-34-20118 и РНФ № 19-74-20110.

%\end{small}

\newpage
\hspace{-10mm}\textbf{\Large Содержание работы}

\vspace{3mm}

\hspace{-10mm}\textbf{\Large Обзор литературы.}
Глава представлена тремя разделами. Первый раздел посвящен описанию морфо-физиологических адаптивных изменений у подземных грызунов. Второй раздел освещает изучение молекулярных адаптаций подземных грызунов и показывает возможность адаптивных изменений как в митохондриальном, так и в ядерном геномах. Третий раздел посвящен подсемейству Arvicolinae, описанию основных подземных представителей этого подсемейства и обзору изучения молекулярных адаптаций у его представителей. 

\vspace{3mm}

\hspace{-10mm}\textbf{\Large Материалы и методы.}

\vspace{3mm}

Работа включает в себя поиск следов отбора в пределах одного митохондриального гена \textit{CYTB}, среди полных митохондриальных геномах и 112 ядерных генах. Количество видов, взятых в анализ на каждом из этапов, отличается из-за доступности материала и количества открытых данных. Сиквенсы были получены из материала коллекции ЗИН РАН и из базы данных GenBank (https://www.ncbi.nlm.nih.gov/genbank/). В каждом случае мы использовали максимально репрезентативную филогенетическую выборку из доступных на момент исследования видов для адекватного таксономического контекста.

\textbf{Выделение ДНК.} Образцы мышечной и кожной ткани хранили в 96\% этаноле при -20 $^\circ$С в коллекции тканей и ДНК лаборатории эволюционной геномики и палеогеномики ЗИН РАН. \underline{Для амплификации методом Сенгера} геномную ДНК выделяли с использованием стандартного протокола солевой экстракции (\cite{Miller1999}). Выделение ДНК проводили в специально оборудованной лаборатории эволюционной геномики и палеогеномики ЗИН РАН. \underline{Для секвенирования методом NGS} (next generation sequencing) геномную ДНК экстрагировали с помощью Diatom DNA Prep 200 (Isogen, Россия). Ультразвуковая фрагментация проводилась с использованием сфокусированного ультразвукового прибора Covaris S220 (Covaris). Фрагментированную ДНК очищали и концентрировали с использованием гранул AMPure XP (Beckman-Coulter). Концентрацию оценивали флуориметром Qubit (Thermo Fisher). Выделение ДНК проводили в центре коллективного пользования в области геномики Сколтеха (https://www.skoltech.ru/research/en/shared-resources/gcf-2/).

\textbf{Выделение РНК}. Образцы смешанных тканей для получения транскриптомов хранили в фиксаторе intactRNA (Евроген, Россия) в коллекции тканей и ДНК лаборатории эволюционной геномики и палеогеномики ЗИН РАН. Выделение РНК производилось с использованием набора RNeasy mini kit (Qiagen) по протоколу для клеток животных (animal cells/spin) со следующими модификациями: 1) на шаге 4 добавляли 0,5 объема 96\% EtOH; 2) после добавления спирта помещали пробирку в термостат на 37 $^\circ$С на 2 минуты; 3) элюция проводилась в 30 мкл воды, очищенной от РНКаз. Гомогенизацию проводили растиранием пестиком в ступке с жидким азотом. Целостность РНК (RIN) определяли на приборе Bioanalyzer 2100 (Agilent). Для дальнейшей работы использовали образцы с RIN не менее 7. Выделение РНК проводили в центре коллективного пользования в области геномики Сколтеха (https://www.skoltech.ru/research/en/shared-resources/gcf-2/).

\textbf{Амплификация отдельных генов методом полимеразной цепной реакции (ПЦР)}. В первой части работы для лучшего разрешения филогенетического дерева Arvicolinae наряду с самим геном \textit{CYTB} были использованы и семь ядерных: ген рака груди 1 (\textit{BRCA1}), экзон 11; ген рецептора гормона роста (\textit{GHR}), экзон 10; фрагмент гена лецитин-холестерин-ацилтрансферазы (\textit{LCAT}), экзоны 2-5 и интроны 2-4; ген белка-супрессора опухолей (\textit{PT53}), экзоны 5-7 и интроны 5-6; ген интерфоторецепторного ретиноид-связывающего белка (\textit{IRBP}); ген фактора фон Виллебранда (\textit{vWF}), экзон 28; и ген кислой фосфатазы типа V (\textit{Acp5}), экзоны 2 и 3. Условия амплификации использовали без изменений (\cite{Lebedev2007}; \cite{Ohdachi2001}; \cite{Bannikova2013}; \cite{Abramson2009}; \cite{Petrova2016}; \cite{Poux2006}). 

%ПЦР-очистку проводили с использованием набора Omnix («Омникс», Россия). ПЦР-продукты секвенировали в обоих направлениях с использованием ABI BigDye версии 3.1. на автоматическом капиллярном секвенаторе Genetic Analyzer 3130 (Applied Biosystems) в компании Евроген (https://evrogen.ru/). 

\textbf{Секвенирование}. \underline{Для секвенирования методом Сенгера} очистку ПЦР-продуктов проводили набором Omnix («Омникс», Россия). ПЦР-продукты секвенировали в обоих направлениях с использованием ABI BigDye версии 3.1. на автоматическом капиллярном секвенаторе Genetic Analyzer 3130 (Applied Biosystems) в компании Евроген (https://evrogen.ru/). \underline{Получение чтений для сборки митохондриальных геномов.} При подготовке библиотек ДНК для высокопроизводительного секвенирования (NGS) был использован набор NEBNext Ultra II DNA Library Prep Kit for Illumina (New England Biolabs). Приготовление библиотек производилось по протоколу со следующими изменениями: 1)  очистка на магнитных частицах после лигирования проводилась по пункту 3В (without size selection) в соотношении объем образца к объему магнитных частиц -- 1:0,9; 2) число циклов в ПЦР -- 10. Полученные продукты были очищены и концентрированы при помощи магнитных частиц в соотношении объем образца к объему магнитных частиц -- 1:0,9. Элюция проводилась в 20 мкл бидистиллированной воды. Концентрация образцов измерялась на флуориметре Qubit. Проверка качества полученных библиотек проводилась при помощи Bioanalyzer 2100 Agilent с помощью набора DNA High Sensitivity kit. Секвенирование проводили на приборе HiSeq4000 (Illumina) со следующими параметрами: длина чтения -- 75 п.н., парные чтения. Демультиплексирование и перевод данных в формат fastq проводили с помощью программы bcl2fastq2. Секвенирование проводили в центре коллективного пользования в области геномики Сколтеха (https://www.skoltech.ru/research/en/shared-resources/gcf-2/). \underline{Получение чтений для сборки транскриптомов}. Для выделения полиА-РНК из общей фракции РНК использовался совмещенный протокол наборов NEBNext Poly(A) mRNA Magnetic Isolation Module и NEBNext Ultra II Directional RNA Library Prep Kit for Illumina (https://international.neb.com/protocols/). Подготовка проводилась со следующими модификациями:  1) фрагментация образцов проводилась при 94$^\circ$С 10 минут; 2) синтез первой цепи проводили со следующими параметрами: 25$^\circ$С -- 10 минут, 42$^\circ$С -- 30 минут, 70$^\circ$С -- 15 минут; 3) очистку и концентрирование ДНК проводили магнитными частицами Ampure XP. Элюцию делали бидистиллированной водой;  4) число циклов в ПЦР -- 10 или 15, в зависимости от исходной концентрации РНК. Концентрация образцов измерялась на флуориметре Qubit. Проверка качества полученных библиотек проводилась при помощи Bioanalyzer 2100 Agilent с помощью набора DNA High Sensitivity kit. Секвенирование проводили на приборе HiSeq4000 (Illumina) со следующими параметрами: длина чтения -- 75 п.н., парные чтения. Демультиплексирование и перевод данных в формат fastq проводили программой bcl2fastq2. Секвенирование проводили в центре коллективного пользования в области геномики Сколтеха (https://www.skoltech.ru/research/en/shared-resources/gcf-2/).

\textbf{Сборка и аннотация.} \underline{Митохондриальные геномы.} Качество сырых чтений оценивалось с помощью FastQC (\cite{Andrews2010}). Очистка от адаптеров и фрагментов с низким качеством проводилась программой Trimmomatic (\cite{Bolger2014}). В работу брали только чтения качеством выше 28-29. Поскольку выделение митохондриальной ДНК \textit{Hyperacrius fertilis} True, 1894 производилось из коллекционных образцов прошлого века, это могло сказаться на качестве отсеквенированных последовательностостостей. Для оценки качества ридов для этого вида дополнительно использовали программу mapDamage 2.0 (\cite{Jonsson2013}). Сборка \textit{de novo} митохондриальных геномов осуществлялась программой plasmid SPAdes (\cite{Bankevich2012}) с настройками по умолчанию. Для дальнейшей аннотации отбирали контиг с наибольшим сходством размера с митохондриальной ДНК для млекопитающих (примерно 16 т.п.н). Контиги аннотировали в веб-сервере MITOS (\cite{Bernt2013}) с настройками по умолчанию и с учетом митохондриального генетического кода позвоночных. Границы генов были проверены при сравнении с 21 опубликованной последовательностью митогенома Arvicolinae. Все позиции с низким качеством и покрытием были заменены на N вручную. Последовательности белок-кодирующих генов проверяли на содержание преждевременных стоп-кодонов вручную. \underline{Tранскриптомы} собирали пакетом Trinity (\cite{Grabherr2011}) с настройками по умолчанию. Поиск кодирующих участков проводили программой Transdecoder (https://github.com/TransDecoder/). Последовательности очищали от химерных генов программой DIAMOND (\cite{Buchfink2015}), беря для сравнения белковую базу NCBI (nr). Ортологичные гены определяли с помощью Proteinortho (\cite{Lechner2011}) и фильтровали в среде R 3.4.4 (\cite{RCoreTeam2017}), оставляя только гены в одной копии. 

\textbf{Выравнивание}. Последовательности гена \textit{CYTB} и полных митохондриальных геномов были выровнены с помощью программы Mauve 1.1.1 (\cite{Darling2004}), реализованной как плагин Geneious Prime 2019.1 (https://www.geneious.com). Конкатенированная последовательность 13 белок-кодирующих митохондриальных генов была отдельно выровнена MAFFT 7.222 (\cite{Katoh2014}). Ядерные ортологичные гены по отдельности были выровнены программой prank (http://wasabiapp.org/software/prank/) с учетом триплетности кодирующей последовательности. Общие для всех фрагменты выравнивания редактировали вручную. Выравнивания конкатенировали в единую последовательность скриптом на языке программирования Python 3.

\textbf{Оценка нуклеотидного состава митохондриальных геномов}. Базовый нуклеотидный состав (процентное содержание каждого из нуклеотидов) рассчитывали в Geneious Prime 2019.1. Смещение в нуклеотидном составе (GC-skew) оценивали как $CG_{skew} = \frac{C - G}{C + G}$ (\cite{Arabi2010}; \cite{Hassanin2005}) с помощью пакета BioSeqUtils в BioPython (\cite{Cock2009}), Python 3. 

\textbf{Филогенетическая реконструкция}. \underline{По гену \textit{CYTB} и семи ядерным генам.} Для анализа изменчивости \textit{CYTB} было построено филогенетическое дерево на основе гена \textit{CYTB} и семи ядерных генов (\textit{BRCA1}, \textit{GHR}, \textit{LCAT},\textit{PT53}, \textit{IRBP}, \textit{vWF}, \textit{Acp5}). Наилучшее соответствие моделей замены для каждого гена оценивали с помощью Treefinder (\cite{Jobb2004}) в соответствии со скорректированным информационным критерием Акаике (AICc). Байесовский анализ был проведен в MrBayes 3.2.6 (\cite{Ronquist2012}). \underline{По митохондриальным геномам.} Для оценки отбора была создана филогенетическая реконструкция для 57 видов подсемейства полевочьих и 6 видов внешней группы. Реконструкция проводилась в программе MrBayes 3.2.2, используя 13 белок-кодирующих генов (11 417 п.н.) и следующие параметры: nst=mixed, гамма-распределение скоростей замен между сайтами, деление на партиции по генам. Каждый анализ начинался со случайных деревьев, два независимых прогона с 4 Марковскими цепями Монте-Карло (MCMC) выполнялись для 5 миллионов поколений, с выборкой каждого 1000-го поколения. Стандартные отклонения разделенных частот были ниже 0,01, потенциальные коэффициенты уменьшения масштаба были равны 1,0, а сходимость оценивали с помощью статистики ESS в Tracer v1.6 (\cite{Rambaut2014}). Консенсусное дерево было построено на основе деревьев, отобранных после 25\% отжига. Дерево визуализировали в программе FigTree v1.4 (http://tree.bio.ed.ac.uk/software/figtree/). \underline{По ортологичным ядерным генам.} Конкатенированное выравнивание ядерных ортологичных генов было очищено от неинформативных сайтов программой Gblocks (\cite{Castresana2000}). Деревья построены в программе RAxML (\cite{Stamatakis2014}) с 500 репликами бутстрепа. Конценсусное дерево выведено методом Majority Rule. 

\textbf{Оценка аминокислотных замен и их распределения}. \underline{Цитохром \textit{b}.} Физико-химические изменения были обнаружены в TreeSAAP v3.2 (\cite{Woolley2003}). Достоверные значения (категории 6-8, P <0,001; \cite{McClellan2001}) были приняты как признак сильного изменения функции белка. Распределение синонимичных и несинонимичных замен было рассчитано на каждом участке гена \textit{CYTB} отдельно и объединены по координатам доменов. Достоверность частоты замен оценивали с помощью точного критерия Фишера и поправки на множественное сравнение методом Холма. Все расчеты проводились в статистической среде R. Координаты доменов были получены на сайте UniProt по гену \textit{CYTB} \textit{Mus musculus} Linnaeus, 1758: https://www.uniprot.org/uniprot/P00158. Частоты использования аминокислот для каждой позиции определяли с использованием данных, которые включали все имеющиеся в базе данных GenBank сиквенсы по взятым в анализ видам на август 2020 года. Аминокислотные паттерны были рассчитаны с использованием скрипта на Python 3. Тест Фишера для сравнения частот проводили с учетом неравных размеров выборки, поправку на множественное сравнение проводили методом Холма в статистической среде R. \underline{Митохондриальные геномы и транскриптомы.} Оценку замен проводили программой ProtParCon (github.com/iBiology/ProtParCon). Поиск аминокислот, характерный только для подземных грызунов, осуществлялся с помощью рукописного скрипта на Python 3. Оценку достоверности обнаруженных замен делали функцией ProtParCon с дальнейшей поправкой Холма на множественное сравнение в статистической среде R. 

\textbf{Оценка уровня отбора}. Количество несинонимичных (\textit{dN}) и синонимичных (\textit{dS}) замен, а также $\omega$ (их соотношение) были рассчитаны несколькими способами. В каждом случае уровень и направление отбора были оценены независимо для подземного вида (или рода) и его филогенетически близких наземных видов. Оценку проводили: (1) с использованием codeml, реализованного в ete-toolkit (\cite{Huerta-Cepas2016}). Значения 999 и 0,001 были расценены как ошибки. Для сравнения различных моделей рассчитаны likelihood-ratio test (LRT). (2) Программой RELAX (\cite{Wertheim2015}) и (3) алгоритмом aBSREL (\cite{Smith2015}).     

%\begin{figure}[h!]
%	\begin{center}
%		\includegraphics[width=\textwidth]{separate_mito_col}
%	\end{center}
%	\caption{Филогенетические деревья, использованные для оценки отбора отдельных таксонов. Подземные виды обозначены цветом.}
%	\label{tree_mito}
%\end{figure}

\textbf{Моделирование и визуализация третичной структуры белка \textit{cytb}}. За основу брали модель кристаллической структуры  \textit{Bos taurus} Linnaeus, 1758 с разрешением 2,4 Å (1NTM; \cite{Gao2003}). Программа modeller 9.22 (\cite{Webb2016}) использовалсь для создания структур комплекса протоколом автомоделирования с настройками по умолчанию. Замены были проанализированы визуально в PyMOL v.2.0 (Schrödinger, LLC). Трансмембранные участки комплекса оценивали с помощью веб-сервера OPM (\cite{Lomize2012}). 

%Моделирование основано на гомологии структуры \textit{Lemmus sibiricus} Kerr, 1792 и \textit{Ellobius lutescens} Thomas, 1897.

\textbf{Предсказание сайтов фосфорилирования}. Для оценки статуса фосфорилирования были использованы веб-серверы NetPhos 3.1 Server (http://www.cbs.dtu.dk/services/NetPhos/, \cite{Blom2004}) и GPS 5.0 (http://gps.biocuckoo.cn/online.php, \cite{Xue2011}). Оценка вероятности NetPhos 3.1 опеределяется в диапазоне [0,00-1,00]. Значения выше 0,50 указывают на наличие форсорилирования. 

\textbf{Репозиторий.} Все рукописные скрипты доступны в репозитории:  https://github.com/ZaTaxon/work\_skripts.

\vspace{3mm}

\hspace{-10mm}\textbf{\Large Поиск следов отбора в гене \textit{CYTB}}

\vspace{3mm}

На первом этапе мы использовали последовательности 62 представителей Arvicolinae. Среди них были почти все филогенетически неродственные подземные виды: представители рода \textit{Ellobius} Fischer, 1814, \textit{Prometheomys schaposchnikowi} Satunin, 1901, \textit{Lasiopodomys mandarinus}, \textit{Terricola subterraneus} de Selys-Longchamps, 1836 и \textit{Mynomes pinetorum} Le Conte, 1830.  

\textbf{Оценка частоты несинонимичных замен.} Анализ распределения несинонимичных замен по сайтам показал три позиции с достоверно более высокими значениями частот замен у подземных видов: 4, 237 и 241. Кроме того, в позиции 236 достоверно повышена частота синонимичных замен. Сравнение распределение замен по целым доменам выявило достоверные различия в мембранных доменах 1, 2, 5 и 9 и трансмембранных доменах 5 и 7 для несинонимичных замен; мембранном домене 6 и трансмембранном домене 5 для синонимичных замен.

%\begin{figure}[h!]
%	\begin{center}
%		\includegraphics[width=0.7\textwidth]{cyt_domains}
%	\end{center}
%	\caption{Частоты распределения несинонимичных (A) и синонимичных (B) замен по доменам гена \textit{CYTB} у подземных и назменых грызунов. Достоверные различия между частотами обозначены звездочками(*): * -- p.value < 0,05; ** -- p.value < 0,01; *** -- p.value < 0,001 }\label{Cyt_Dom_fig}
%\end{figure} 

%С помощью TreeSAAP мы получили список всех значимых аминокислотных замен (категории 6-8) для изучаемых видов в гене \textit{CYTB}. Из него мы выбрали те, которые встречаются по крайней мере у трех из подземных видов, но отсутствуют у всех наземных видов. 

\begin{figure}[h!]
	\begin{center}
		\includegraphics[width=0.9\textwidth]{Cytb_tree_colour_ed}
	\end{center}
	\caption{Филогенетическое дерево взятых в анализ видов. Указаны аминокислотные замены, характерные для подземных видов в гене \textit{CYTB}. Подземные виды отмечены цветом.}
	\label{PhyloTree}
\end{figure}


\textbf{Поиск параллельных аминокислотных замен.} Мы нашли три замены в гене \textit{CYTB}: Ser57Pro, Asp214Asn и Ile338Val (рис. \ref{PhyloTree}). Замена Asp214Asn была обнаружена также и у специализированных подземных грызунов из других семейств. Замена серина на пролин в остатке 57 у подземных грызунов потенциально удаляет сайт фосфорилирования. Сервер NetPhos 3.1 предсказал фосфорилирование киназой \textit{CDC2} с оценкой 0,518. GPS 5.0 выявил киназы \textit{AGC}, \textit{PKN} и \textit{PKN1}. Предсказания типа киназы не согласуются друг с другом, однако все прогнозы указывают на высокую вероятность фосфорилирования этого сайта. Нуклеотидная замена в кодоне 338 (ATT > GTT) была обнаружена как вероятный патоген в базе данных ClinVar и связана с раковыми процессами: www.ncbi.nlm.nih.gov/clinvar/variation/143898/. Согласно третичной структуре белка цитохрома (рис. \ref{CytStructure} A), Ser57 обращен к межмембранному пространству митохондрий. Он расположен на неструктурированном сегменте петли, охватывающем остатки 54–60. Эта петля контактирует с той же петлей на втором мономере цитохрома \textit{bc1} в комплексе (рис. \ref{CytStructure} B). Замена Asp214Asn находится на петле, обращенной к матрице митохондрии. Он контактирует с N-концом субъединицы VII комплекса убихинол-цитохром с редуктазы III (UQCRQ) (рис. \ref{CytStructure} C). Замена Ile338Val находится на границе раздела $\alpha$-спиралей в трансмембранной области комплекса (рис. \ref{CytStructure} D). Смоделированная структура показывает, что эта замена благоприятствует другому ротамеру Ile350, который соседствует с остатком 58 UQCRQ.



%Те же методы не предсказывали фосфорилирование для Asp214Asn, и, насколько нам известно, ни Ile, ни Val в позиции 338 не могут быть фосфорилированы.

\begin{figure}[h!]
	\begin{center}
		\includegraphics[width=0.6\textwidth]{fig структура}
	\end{center}
	\caption{Структурная модель замен в комплексе цитохрома \textit{bc1}. \textbf{A.} Обзор гомодимера цитохрома \textit{bc1}. Цитохром Б --- голубой, UQCRQ - пурпурный. Второй мономер окрашен в желтый цвет. Места замены выделены кружками. IMS --- межмембранное пространство. \textbf{B.} Увеличенные структуры \textit{E. lutescens} и \textit{L. sibiricus}, показывающие замену Ser57Pro. Модель \textit{E. lutescens} голубая, \textit{L. sibiricus} --- белая. \textbf{C.} Замена Asp214Asn и его взаимодействие с N-концом UQCRQ (пурпурный) \textbf{D.} Замена Ile338Val и соседняя цепочка UQCRQ (пурпурный)}
	\label{CytStructure}
\end{figure}

\textbf{Оценка уровня отбора}. Оценка значений $\omega$ показала общую тенденцию к ослаблению уровня отбора у подземных грызунов при сравнении их с филогенетически близкими наземными видами. Достоверные отличия получены с использованием программы codeml для видов рода \textit{Ellobius}, \textit{L. mandarinus} и \textit{T. subterraneus}. Почти у всех подземных видов наблюдаются более высокие значения $\omega$ по сравнению наземными, за исключением \textit{T. subterraneus}. Эта разница варьирует от одного (для \textit{Microtus pinetorum}) до пяти раз для \textit{Lasiopodomys mandarinus}. Анализ алгоритмом aBSREL не показал свидетельств эпизодического отбора. Результаты программы RELAX подтвердили изменения в уровне отбора у подземных грызунов. Так, коэффициент отбора K для трех подземных представителей (\textit{Ellobius sp.}, \textit{L. mandarinus} и \textit{P. schaposchnikowi}) показал значения < 1, что может указывать на процесс ослабления отбора. 

\vspace{3mm}

\hspace{-10mm}\textbf{\Large Поиск следов отбора в митохондриальных геномах}

\vspace{3mm}

\textbf{Характеристика собранных митохондриальных геномов.} Всего в лаборатории эволюционной геномики и палеогеномики ЗИН РАН было собрано 34 новых митохондриальных генома представителей Arvicolinae. Анализ повреждений ДНК с помощью mapDamage для древнего образца \textit{Hyperacrius fertilis} показал низкое значение дезаминирования (рис. \ref{MapDamage}). Неправильное включение от C до T варьировалось от 12.09 \% до 17.94 \%, от G до A -- от 12.84 \% до 17.49 \%. Уровни ошибочного включения сравнимы со всеми другими вариантами замен, а также аналогичными значениям из статей (\cite{Molto2017}). Нами не было обнаружено каких-либо структурных изменений в порядке митохондриальных генов или их количестве, которые отличали бы подземных грызунов от наземных сестринских видов. Результаты сравнения нуклеотидных составов показали увеличение среднего значения \% GC и уменьшение GC-skew у подземных полевочьих, но разница оказалась недостоверная.

\begin{figure}[h!]
	\begin{center}
		\includegraphics[width=0.8\textwidth]{MapDamage Hyper}
	\end{center}
	\caption{Замены, которые могут являться следствием дезаминирования нуклеотидов, помечены цветом: замены гуанина на аденин (G > A, голубой цвет) и цитозина на тимин (C > T, красный цвет). Остальные варианты замен отмечены серым.}\label{MapDamage}
\end{figure}

\textbf{Анализ частоты несинонимичных замен.} %Для оценки количества замен в митохондриальных геномах использовали программу ProtParCon. Если подсчитать в целом количество несинонимичных замен, нормировав их на количество взятых в анализ видов, то наблюдается следующее: 
Доля несинонимичных замен в митохондриальных геномах подземных грызунов оказалось почти в два выше, чем у наземных. Отдельно в каждом гене эта тенденция повторяется с достоверной разницей.

\textbf{Поиск параллельных аминокислотных замен}. Анализ выявил параллельные замены \textit{COX1} Met73Ile, \textit{COX3} Ile121Val, \textit{ND5} Phe446Leu, \textit{CYTB} Thr56Ser, \textit{CYTB} Ile338Val, \textit{CYTB} Ala357Thr. При дальнейшей статистической проверке они оказались недостоверными. 

\textbf{Оценка уровня и направления отбора}. При анализе программой codeml у представителей рода \textit{Ellobius} разница в уровне отбора наблюдается во всех митохондриальных генах, кроме \textit{ND2,3,5,6} и \textit{COX3} при сравнении с наземными сестринскими видами. Несколько генов с достоверно отличающимся уровнем обнаружено также у \textit{L. mandarinus}: \textit{COX3} и \textit{CYTB}. Один ген найден для \textit{P.\ schaposchnikowi} -- \textit{COX3}. У оставшихся подземных грызунов не обнаружено генов с достоверным отличием. Хотя значения $\omega$ существенно различались в зависимости от генов и анализируемых видов, они не превышали единицы и были выше для подземных видов, чем для наземных. Используя алгоритм aBSREL, мы обнаружили следы эпизодического положительного отбора в гене \textit{COX2} для \textit{E. lutescens} Thomas, 1897 и в двух генах \textit{P. schaposchnikowi}: \textit{ATP8} и \textit{ND5}. Анализ RELAX подтвердил изменения в уровне отбора подземных грызунов. Семь генов для представителей \textit{Ellobius} оказалось с K-значениями < 1: \textit{ATP6}, \textit{COX1}, \textit{COX3}, \textit{CYTB}, \textit{ND1}, \textit{ND2} и \textit{ND4} и столько же для \textit{P. schaposchnikowi} при сравнении с <<первой радиацией>> Arvicolinae: \textit{COX1}, \textit{COX3}, \textit{ND2}, \textit{ND4} и \textit{ND5}. Ген \textit{COX3} \textit{P. schaposchnikowi} также достоверно отличается от видов рода \textit{Cricetulus}. Список генов \textit{L. mandarinus} с достоверными отличиями включает всего три: \textit{COX1}, \textit{COX3} и \textit{CYTB}. У оставшихся подземных грызунов гены с достоверным ослаблением отбора не обнаружены. Многие гены выявлены в анализах у нескольких подземных полевочьих одновременно. Так, для генов \textit{COX3} и \textit{COX1} наблюдается ослабление отбора у видов рода \textit{Ellobius}, \textit{P. schaposchnikowi} и \textit{L. mandarinus}. Некоторые гены были обнаружены дважды для видов \textit{Ellobius} и \textit{P. schaposchnikowi} (например, \textit{ND2} и \textit{ND4}) или представителей \textit{Ellobius} и \textit{L. mandarinus} (\textit{CYTB}).

\vspace{3mm}

\hspace{-10mm}\textbf{\Large Поиск следов отбора в транскриптомах.}

\vspace{3mm}

\textbf{Сборка транскриптомов}. В ходе работы нами собрано 17 транскриптомов: сырые риды для 10 видов были полученных нами лично в рамках проекта РНФ и 7 взяты из открытой базы данных SRA, также мы использовали уже собранные транскриптомы для \textit{Microtus ochrogaster} Wagner, 1842 и \textit{Cricetulus griseus} Milne-Edwards, 1867. Статистика собранных транскриптомов показала, что все из них можно использовать в дальнейшем анализе. В них мы нашли 112 универсальных однокопийных ортологов. 
%После очистки собранных транскриптомов мы приступили к поиску универсальных однокопийных ортологов, которые присутствуют в одной копии у всех взятых в анализ видов. На этом этапе мы добавили к нашим данным два уже собранных и выложенных в базе данных Genome транскриптома: \textit{Microtus ochrogaster} Wagner, 1842 и \textit{Cricetulus griseus}.

\textbf{Анализ частоты несинонимичных замен}. Сравнение частот несинонимичных замен между подземными и наземными видами не показало достоверных различий. Также нами не было обнаружено отдельных генов  или позиций, в которых частоты будут достоверно различаться. 

\textbf{Оценка параллельных аминокислотных замен}. Нами были обнаружены достоверные замены в генах \textit{Rad23b} -- Thr121Ala и \textit{Pycr2} -- Ala314Thr.

\textbf{Оценка уровня отбора}. Мы провели поиск следов отбора в найденных нами ортологичных генах независимо для каждой линии подземных грызунов, однако генов с достоверными отличиями обнаружено не было.

\vspace{3mm}

\hspace{-10mm}\textbf{\Large Изменения на молекулярном уровне у полевочьих, перешедших к подземному образу жизни}

\vspace{3mm}

В нашей работе получены данные по изменению уровня отбора в митохондриальных генах подземных грызунов, а также обнаружены параллельные замены в ядерных генах. Помимо этого, мы сравнивали базовые характеристики митохондриальных геномов: GC-состав, его смещение, количество и порядок генов. 

\textbf{Изменение нуклеотиного состава митохондриальных геномов.} Оценка GC-состава показывает повышение количества этих нуклеотидов у подземных грызунов. Несмотря на то, что разница недостоверна, это может указывать на адаптивные следы в митохондриальном геноме. Генный состав митохондрий среди всех изученных видов остается стабильным и неизменным. Переход к подземному образу жизни не повлек за собой изменение в порядке генов или удвоению каких-то конкретных блоков.


\textbf{Ослабление уровня отбора.} Исторически изучение адаптивности митохондриальных генов началось с гена \textit{CYTB}. Начатые с трудов Адрюса (\cite{Andrews1998}), и в последующем повторенные целым рядом исследователей (\cite{Tomasco2014}; \cite{DaSilva2009}; \cite{DiRocco2006}; \cite{Shao2015}), работы свидетельствовали о наличии признаков положительного отбора при эволюции гена \textit{CYTB}. Да Сильва с коллегами (\cite{DaSilva2009}) обнаружили достоверную разницу уровня отбора в подземных линиях (\textit{Ctenomys} Blainville, 1826, \textit{Spalacopus} Wagler, 1832, семейства Geomyidae и Bathyergidae) по сравнению с их назмеными близкородственными видами. Результаты Цанга с коллегами предполагали, что эволюция гена \textit{CYTB} цокора \textit{Eospalax cansus} G. M. Allen, 1938 также может определяться изменением уровня отбора. Более того, распределение несинонимичных мутаций указывало на значительные изменения в последовательности \textit{CYTB} у животных, которые столкнулись с более тяжелой гипоксией из-за большей высоты и более холодного и сухого климата, чем другие митохондриальные линии (\cite{Zhang2013a}). Это предполагает связь между эволюцией гена \textit{CYTB} и колонизация гипоксической среды. 

%Дальнейшие исследования показали, что, несмотря на сильные функциональные ограничения, митохондриальная ДНК в целом может подвергаться положительному направленному отбору, например, в случаях, когда требуется энергоемкий образ жизни и/или есть ограничения в доступности кислорода (\cite{Tomasco2011}; \cite{Shen2010}; \cite{Blier2001}).

В нашем исследовании гена \textit{CYTB} мы также обнаружили ослабление отбора у некоторых подземных грызунов и повышенное значение $\omega$ у подземных грызунов по сравнению с наземными. Восемь белковых доменов обладают повышенной частотой замен у подземных видов, тоже наблюдается в нескольких нуклеотидных позициях. Эти результаты согласуются с гипотезой о том, что колонизация подземной ниши способствует положительному отбору в генах митохондриальной ДНК. Признаки усиления положительного отбора были выявлены как у четырех видов \textit{Ellobius}, так и у \textit{L. mandarinus}. Мы обнаружили достоверную разницу в значениях $\omega$, анализируя \textit{T. subterraneus} с другими видами \textit{Terricola} и \textit{Microtus} Schrank, 1798, но в этом случае значение $\omega$, наоборот, было меньше для \textit{T. subterraneus}. Данные анализа RELAX согласуются с результатами codeml и показывают, что ген \textit{CYTB} у подземных грызунов подвержен ослаблению уровня отбора.

Позже указанные для гена \textit{CYTB} наблюдения были подтверждены для всех митохондриальных белок-кодирующих генов. Подземные виды южноамериканских туко-туко (\textit{Ctenomys}) и родственные восьмизубы (\textit{Spalacopus}) в работе Томаско и Лесса (\cite{Tomasco2011}) показали достоверно более высокие значения $\omega$ по сравнению наземными видами в 11 из 13 митохондриальных генов. Конвергентные изменения были также обнаружены между изученными подземными видами и другими млекопитающими, адаптированными к гипоксии. Таверс и коллеги выявили достоверное ослабление отбора в большинстве митохондриальных генов подземных африканских землекопов, туко-туко и восьмизубов (\cite{Tavares2018}). 

Результаты, полученные нами при исследовании всех белок-кодирующих митохондриальных генов, также говорят о вовлеченности всего митохондриального генома в адаптивный процесс. Метод codeml показал повышение уровня отбора у подземных грызунов. Больше всего различий обнаружено у представителей рода \textit{Ellobius}: разница в отборе наблюдается во всех митохондриальных генах, кроме \textit{ND2,3,5,6} и \textit{COX3}. Два гена под отбором обнаружено у \textit{L. mandarinus}: \textit{COX3} и \textit{CYTB} и один ген у \textit{Prometheomys} -- \textit{COX3}. При анализе RELAX у представителей рода \textit{Ellobius} ослабление отбора наблюдается в 8 генах, у \textit{Prometheomys} --- в пяти (при сравнении с первой радиацией) и одном при сравнении с хомяками. Для \textit{L. mandarinus} разница обнаружена в трех генах. Мы обнаружили гены, которые показывают достоверные различия для более чем одного проанализированного подземного вида, и гены, обнаруженные в более чем в одном анализе.  Так, гены \textit{CYTB} и \textit{COX3} продемонстрировали более высокие значения $\omega$ одновременно у видов \textit{P. schaposchnikowi} и \textit{L. mandarinus} (\textit{COX3}) и \textit{L. mandarinus} и \textit{Ellobius} (\textit{CYTB}) при оценке уровня отбора методом codeml. Гены \textit{COX3}, \textit{COX1}, \textit{ND2} и \textit{ND4} демонстрируют ослабление отбора согласно программе RELAX по крайней мере для двух видов. Гены \textit{COX1} и \textit{COX3} были обнаружены как наиболее изменчивые при сравнении результатов нескольких программ. 

Набор генов с разницей в уровне отбора у подземных и наземных грызунов частично коррелирует со скоростью их эволюции. Скорости изменчивости среди семейств митохондриальных генов распределяются следующим образом: \textit{ATP}> \textit{ND}> \textit{CYTB}> \textit{COX} по Лопез (\cite{Lopez1997}). По нашим результатам, оба гена \textit{ATP} показывают достоверную разницу в уровне отбора для представителей рода \textit{Ellobius}. Кроме того, они выявляются в других анализах: RELAX (\textit{ATP6} для рода \textit{Ellobius}) и aBSREL (\textit{ATP8} для \textit{P. schaposchnikowi}). Гены семейства \textit{ND} показывают неоднородность в результатах. Для генов \textit{ND2}, \textit{ND4} и \textit{ND5} обнаружены достоверные различия для трех из пяти проанализированных подземных видов. Анализ гена \textit{CYTB} подтверждает ослабление отбора для видов \textit{Ellobius} и \textit{L. mandarinus}. Эти результаты повторяют полученные при анализе отдельно гена \textit{CYTB} (\cite{Bondareva2021}). Неожиданный результат получен для генов семейства \textit{COX}: несмотря на свой консерватизм, они показывают достоверные изменения в уровне отбора во всех анализах на том же уровне, что и более вариабельные гены \textit{ND}.

С одной стороны, наши данные согласуются с опубликованным ранее работами и показывают, что митохондриальный геном подземных полевочьих потенциально вовлечен в процесс адаптации к подземному образу жизни. Но с другой, мы видим различие в уровне отбора между разными видами. Потенциально, это может быть связано с уровнем специализации к подземному образу жизни и его стилем. К сожалению, обнаруженная на митохондриальных геномах тенденция к ослаблению отбора не подтвердилась на ядерных данных, и нам не удалось найти гены, уровень отбора в которых достоверно бы различался у подземных и наземных грызунов. Это может быть связано как с более медленными темпами эволюции (\cite{Lin2004}) по сравнению с митохондриальными генами, так и с недостаточным количествов генов, взятых в анализ. В работе Калины Дейвис (\cite{Davies2018}), например, по исследованию адаптаций подземных млекопитающих только в 10\% от всех проанализированных генов (которых было около 8 тыс) обнаружена достоверная разница в отборе между подземными и наземными видами. 


\textbf{Возникновение паралелльных аминокислотных замен.} Поиск параллельных замен показал себя как действенный способ обнаружить гены, которые могут быть вовлечены в адапционные процессы и при этом не меняют уровень или направление отбора (\cite{Zhou2015}, \cite{Sackman2017}, \cite{Davies2018}). При анализе отдельно гена \textit{CYTB} нами было обнаружено три замены, характерные для подземных грызунов: Ser57Pro, Asp214Asn, и Ile338Val. Замены в сайте 57 также обнаружены у африканских слепышей (семейство Bathyergidae) и в 214 -- у африканских слепышей и туко-туко (род \textit{Ctenomys}). Аналогичные замены в сайте 214 были обнаружены у высокогорных подземных цокоров \textit{Eospalax fontanierii} (\cite{Cooper1993}). Выполняя изначальный поиск на одиночных последовательностях (т.е. имея только одну для одного вида), мы получили возможность проверить найденные замены на популяционном уровне в гене \textit{CYTB}. Использование его как основного филогенетического маркера позволило включить в полномасштабный анализ более 6 тыс. сиквенсов. Из трех позиций, обнаруженных при анализе, две подтвердились: Thr56Ser и Ile338Val. В них мы видим достоверное смещение использования аминокислот у подземных грызунов. При анализе митохондриальных генов нами было обнаружено шесть позиций с характерных для подземных грызунов заменами: \textit{COX1} Met73Ile, \textit{COX3} Ile121Val, \textit{ND5} Phe446Leu, \textit{CYTB} Thr56Ser, \textit{CYTB} Ile338Val, \textit{CYTB} Ala357Thr. Все выявленные замены оказались недостоверными при статистической проверке. Нам удалось найти гены с параллельным аминокислотными заменами, которые есть только у подземных грызунов, в ядерных генах: \textit{Erp29}, \textit{Rad23b}, \textit{Hikeshi}, \textit{Zadh2}, \textit{Mrps14}, \textit{Pycr2}, \textit{Ccdc86}, \textit{GTPBP2}, \textit{Snapc2} и \textit{Ttll12}. Последующая проверка показала достоверность замен в \textit{Rad23b} и \textit{Pycr2}. Согласно литературе, обнаруженное нами небольшое число генов с уникальными заменами  является нормальным и ожидаемым. Так, в работе Калины Дейвис (\cite{Davies2018}) было обнаружено всего 35 генов с паралельными заменами у всех проанализируемых видов, а большее количество составляли гены с уникальными заменами (в нашей работе мы не рассматривали эти варианты) и парными среди разных родов. 

Найденные нами ядерные гены с паралельными заменами не выявлялись ранее при изучении молекулярных адаптаций подземных грызунов. Однако, биохимические пути и процессы, в которые они вовлечены, можно связать с этими процессами. Активная работа митохондрий, которая может быть усилена в условиях сниженной концентрации кислорода, потенциально вызывает образование активных форм кислорода, являющихся опасным для клетки разрушающим фактором (\cite{Turrens2003}). Гены \textit{Rad23b} и \textit{Pycr2} (пирролин-5-карбоксилатредуктаза 2) связаны с процессами репарации ДНК (\cite{Pohjoismaki2012}) и реакцией на окислительный стресс (\cite{Kuo2015}), соответственно. Гомолог гена \textit{Rad23} обнаружен при изучении адаптций к засухе у растений, его уровень отбора сильно изменялся в сторону положительного (\cite{Zhang2013b}). Изменение его экспрессии также выявлен при анализе устойчивости к холоду пырея \textit{Thinopyrum intermedium} (\cite{Jaikumar2020}). 

\textbf{Конвергентные адаптации грызунов к подземному образу жизни.} Мы наблюдаем у подземных полевочьих тенденцию проявления молекулярных адаптаций, описанную ранее в литературе, не смотря на ограниченную выборку полученных и проанализированных генов. В первую очередь, у них независимо ослабляется отбор на митохондриальных генах и увеличивается частота несинонимичных замен в целом. Как в митохондриальных, так и в ядерных геных происходят паралелльные аминокислотные замены в физиологически важных генах. 

Анализ митохондриальных геномов и транскриптомов показал общие характеристики для подземных грызунов. Однако, если анализировать каждую подземную линию независимо, видна неоднородность проявления этих признаков. Так, больше всего изменений затронуло род \textit{Ellobius}, а меньше всего -- \textit{Hyperacrius} (рис. \ref{common_trends_all}). Наблюдаемое количество изменений не коррелирует со временем дивергенции вида. Переход к подземному образу жизни у \textit{P. schaposchnikowi} произошел, согласно молекулярным данным, 7 млн лет назад. Переход же к подземному образу жизни у представителей рода \textit{Ellobius} был совершен в плиоцене. Не смотря на это, морфологические изменения представителей этого рода наиболее близки к тем, что наблюдаются у <<модельных>> подземных грызунов семейств Bathyergidae и Spalacidae: выступающие резцы, очень маленькие глаза и изолирование ротового отдела губами (\cite{Gromov1977}). По некоторым предположениям, повышение уровня отбора в митохондриальных генах могло быть следствием низкой эффективной численности популяции подземных грызунов (\cite{Lacey2000}). Однако, факты, что мы видим этот тренд 1) не во всех генах у одного рода и 2) не у всех видов подземных грызунов, скорее противоречит выдвинутой гипотезе.

\begin{figure}[h!]
	\begin{center}
		\includegraphics [width=0.9\textwidth]{genes_mt_and_nucl}
	\end{center}
	\caption{Общие тренды к изменению уровня отбора и возникновению параллельных замен у подземных грызунов подсемейства Arvicolinae. E. -- \textit{Ellobius} species, P.s. -- \textit{P. schaposchnikowi}, L.m. -- \textit{L. mandarinus}, H.f. -- \textit{H. fertilis}, T. -- \textit{Terricola} species.}\label{common_trends_all}
\end{figure}

Самыми стрессовыми проблемами, с которыми сталкиваются подземные грызуны, являются гипоксические / гиперкапнические условия и перегрев в закрытой системе нор (\cite{Lacey2000}). Хотя и слепушонки, и \textit{P. schaposchnikowi} являются высокоспециализированными подземными грызунами, они занимают разные среды обитания, и характеристики их нор, а также методы добычи пищи весьма различны. Типичные места обитания \textit{Ellobius} --- засушливые или полузасушливые ландшафты, такие как степи, пустыни и луга. Эти грызуны населяют различные типы почв, в том числе плотные почвы глинистых пустынь, и создают довольно устойчивые системы узких туннелей для кормления под землей (\cite{Ognev1950}; \cite{Gromov1977}; \cite{Shubin1978}). Таким образом, слепушонки должны справляться с проблемами, с которыми сталкиваются другие действительно подземные млекопитающие (\cite{Lacey2000}). Последнее, в свою очередь, может привести к наблюдаемым нами изменениям в белок-кодирующих генах. Напротив, \textit{P. schaposchnikowi} встречается на кавказских субальпийских высокотравных лугах на высоте 1500-2500 м (\cite{Vereshchagin1959}; \cite{Vorontsov1966}; \cite{Krystufek2005}). Его норы вырыты в рыхлой влажной почве, заполненной камнями, что должно увеличить диффузию газа. Поверхностные кормовые туннели этих полевок кажутся слишком большими и почти вдвое шире, чем можно было бы ожидать от грызунов такого размера (\cite{Vorontsov1966}; личные наблюдения А.В. Сморкачевой). Это дополнительное пространство может предотвратить перегрев, гипоксию и гиперкапнию. Во время кормления полевка высовывается из норы, подбирает растения и затаскивает их в нору для безопасного кормления (\cite{Gambaryan1957}; \cite{Zimina1977}; личные наблюдения А.В. Сморкачевой). Благодаря холодному климату, архитектуре нор и особенностям кормления, \textit{P. schaposchnikowi} может избежать некоторых физиологических проблем, с которыми приходится сталкиваться большинству подземных видов.

У \textit{L. mandarinus}, несмотря на обнаруженные ранее изменения в циркадных ритмах и адатпации к гипокии (\cite{Sun2018}; \cite{Dong2020}), мы наблюдаем не такие сильные изменения: всего три гена с изменением уровеня отбора --- \textit{COX1}, \textit{COX3} и \textit{CYTB}. Тем не менее, мы обнаружили существенные различия у \textit{L. mandarinus} и \textit{Terricola}, несмотря на схожий эволюционный возраст этих таксонов. Это различие также может отражать неравные уровни энергетического и гипоксического стресса, возникающие из-за специфических характеристик стратегии поиска пищи и архитектуры норы. \textit{L. mandarinus} питаются либо под землей, либо зелеными частями растений в непосредственной близости от входа в нору. Кормовые ходы этого вида расположены на глубине 10-30 см (\cite{Smorkatcheva1990}; \cite{Hong2019}), а прямые измерения концентрации газов, температуры и влажности подтвердили, что животные в норе должны столкнуться с гипоксией и гиперкапнией. \textit{Terricola} населяют различные растительные сообщества от широколиственных лесов (\textit{T. subterraneus}) до альпийских лугов (\textit{T. daghestanicus}) (\cite{Aulagnier2018}). Они используют сложную сеть подземных тоннелей и демонстрируют некоторые внешние черты, связанные с подземным образом жизни (например, \cite{Aulagnier2018}; \cite{Mironov2020}). Однако телосложение и повадки \textit{T. subterraneus} (Н.И. Абрамсон и А.В. Сморкачева, неопубликовано), а также характеристики его туннельной системы отличают эту полевку от специализированных подземных грызунов. Пищевые пути этого вида располагаются в самом поверхностном слое почвы (в пределах 5-10 см) или даже непосредственно под опадой листьев (\cite{Mironov2020}). Туннели обеспечивают защиту полевок от неблагоприятных погодных условий и хищников, что объясняет тенденцию видов \textit{Terricola} к снижению скорости базовых метаболических процессов (\cite{Caroli2000}; \cite{Jemiolo1983}; \cite{Schropfer1977}), но их глубина, вероятно, слишком мала, чтобы существенно предотвратить диффузию газа, приводящую к гипоксическим / гиперкапническим условиям внутри. К сожалению, об экологии \textit{T. daghestanicus}, населяющего кавказские альпийские степи и луга, почти ничего не известно. Тот факт, что полевки этого вида находят убежище среди скал (\cite{Krystufek2005}), предполагает, что они не так строго подземные. Наши результаты подтверждают этот факт, поэтому мы не обнаружили никаких изменений уровня отбора в митохондриальных генах этого вида. 

Подземные полевочьи повторяют адаптационный путь других подземных грызунов, показывая схожие признаки. Так, во время многочисленных исследований представителей семейств Bathyergidae и  Spalacidae были обнаружены паралельные замены в абсолютно разных генах. Также во многих генах видно ослабление отбора по сравнению с наземными видами. Этот эволюционный тренд подтверждается не только на подземных грызунах, но и на других подземных млекопитающих --  Talpidae и Chrysochloridae. Обнаруженные адаптации митохондриального генома (увеличение уровня отбора, наличие параллельных замен) совпадают с исследованиями на Octodontidae и Ctenomyidae. Таким образом, полученные нами результаты согласуются с гипотезой о том, что переход к подземному образу жизни стимулирует ослабление уровня отбора. Причем выраженность скорее связана с уровнем специализации вида, нежели с возрастом его появления.

\vspace{3mm}

\hspace{-10mm}\textbf{\large Заключение.}
В заключении кратко сформулированы основные результаты исследования.

\vspace{3mm}

\hspace{-10mm}\textbf{\large Выводы:}

\begin{enumerate}
	
\item В гене \textit{CYTB} обнаружены параллельные аминокислотные замены, характерные для подземных грызунов и показано увеличение частоты несинонимичных замен.

\item В большинстве генов митохондриального генома наблюдается процесс ослабления отбора у подземных грызунов. 

\item При анализе белок-кодирующих генов были выявлены параллельные замены у подземных грызунов, которые вовлечены в разные биохимические процессы. 

\item Количество генов, в которых произошли изменения уровня отбора или возникли параллельные замены, отличаются для разных видов подземных грызунов и связано скорее с уровнем специализации, а не с эволюционным возрастом вида. 

\item Для подземных представителей эволюционно молодого подсемейства полевочьих характерны те же направления молекулярной адаптивной изменчивости, что и для относительно древних специализированных семейств подземных грызунов.
	
\end{enumerate}


\newpage
\begin{small}
\subsection*{Список работ, опубликованных по теме диссертации}
\subsubsection*{В изданиях из перечня ВАК:}

\begin{enumerate}
\item[\textbullet] \textbf{Bondareva O. V.}, Abramson N. I. The complete mitochondrial genome of the common pine vole \textit{Terricola subterraneus} (Arvicolinae, Rodentia) //Mitochondrial DNA Part B. – 2019. – V. 4. – №. 2. – P. 3925-3926;
\item[\textbullet] Abramson N. I., Golenishchev, F. N., Bodrov, S. Y., \textbf{Bondareva, O. V.}, Genelt-Yanovskiy, E. A., \& Petrova, T. V. Phylogenetic relationships and taxonomic position of genus \textit{Hyperacrius} (Rodentia: Arvicolinae) from Kashmir based on evidences from analysis of mitochondrial genome and study of skull morphology //PeerJ. – 2020. – V. 8. – P. e10364;
\item[\textbullet] \textbf{Bondareva O. V.}, Mahmoudi, A., Bodrov, S. Y., Genelt-Yanovskiy, E. A., Petrova, T. V., \& Abramson, N. I. The complete mitochondrial genomes of three \textit{Ellobius} mole vole species (Rodentia: Arvicolinae) //Mitochondrial DNA Part B. – 2020. – V. 5. – №. 3. – P. 2485-2487; 
\item[\textbullet] \textbf{Bondareva O. V.}, Potapova, N. A., Konovalov, K. A., Petrova, T. V., \& Abramson, N. I. Searching for signatures of positive selection in cytochrome b gene associated with subterranean lifestyle in fast-evolving arvicolines (Arvicolinae, Cricetidae, Rodentia) //BMC Ecology and Evolution. – 2021. – V. 21. – №. 1. – P. 1-12;
\item[\textbullet] \textbf{Bondareva O.}, Bodrov S., Genelt-Yanovskiy E., Petrova T., Abramson N. Signatures of selection and adaptation to subterranean lifestyle across the transcriptomes of Arvicolinae (Rodentia, Cricetidae)// FEBS Open Bio. - 2021. - 11:P-01.3-17. doi:10.1002/2211-5463.13205;
\item[\textbullet] Abramson N. I., Bodrov, S. Y., \textbf{Bondareva, O. V.}, Genelt-Yanovskiy, E. A., \& Petrova, T. V.Mitochondrial genome phylogeny of voles and lemmings (Rodentia: Arvicolinae): evolutionary and taxonomic implications //Plos One. – 2021. - 16(11): e0248198;
\item[\textbullet]  \textbf{Bondareva O.}, Genelt-Yanovskiy, E., Petrova, T., Bodrov, S., Smorkatcheva, A., \& Abramson, N.  Signatures of Adaptation in Mitochondrial Genomes of Palearctic Subterranean Voles (Arvicolinae, Rodentia) //MDPI Genes. – 2021. – V. 12. – №. 12. – P. 1945.
\end{enumerate}

\subsubsection*{В сборниках конференций:}
\begin{enumerate}
	
\item[\textbullet] \textbf{Бондарева О.В.}, Петрова Т.В., Бодров С.Ю., Генельт-Яновский Е.А., Абрамсон Н.И. Следы отбора в митохондриальном геноме при адаптации к подземному образу жизни на примере представителей подсемейства полевочьих (Arvicolinae, Cricetidae, Rodentia) // Материалы отчётной научной сессии ЗИН РАН по итогам работ 2019 г. (26--28 октября 2020, Санкт-Петербург). Издательство ЗИН РАН --- С. 11-13;

\item[\textbullet] \textbf{Olga Bondareva}, Semyon Bodrov, Natalia Abramson. Signatures of natural selection in mitochondrial genome of underground rodents. // Abstracts of the international conference on computer biology MCCMB`19 (July 27-30, 2019, Moscow) http://mccmb.belozersky.msu.ru/2019/index.html;

\item[\textbullet] \textbf{Bondareva O.}, Potapova N., Abramson N. Сan substitutions in mitochondrial protein sequence have adaptive signal? Сase study of voles and lemmings, subfamily Arvicolinae, Rodentia // VII International Congress and Associate Symposiums of Vavilov Society of Geneticists and Breeders on the 100th anniversary of the department of genetics of Saint Petersburg State University (June 18-22, 2019, Saint Petersburg, Russia). Book of abstracts. - P. 140.;

\item[\textbullet] \textbf{Olga V. Bondareva}, Artem Kasianov, Nataliya Abramson. Family-specified direction of selection in underground rodents // 6th International Conference of Rodent Biology and Management and 16th Rodens et Spatium (Potsdam, Germany, 3-7 September 2018), Book of Abstracts - P. 184 / 10.5073/jka.2018.459.000;

\item[\textbullet] Kristina V. Kuprina, \textbf{Olga V. Bondareva}, Antonina V. Smorkatcheva, Nataliya Abramson, Svetlana A. Galkina. Multiple mitochondrial pseudogenes in the nuclear genome in two species of mole voles (\textit{Ellobius}, Cricetidae) // 6th International Conference of Rodent Biology and Management and 16th Rodens et Spatium (Potsdam, Germany, 3-7 September 2018), Book of Abstracts - P. 148 / 10.5073/jka.2018.459.000/;

\item[\textbullet] \textbf{Bondareva O.}, Kasianov A., Abramson N. Do rodent species adopt to underground lifestyle by different ways? // The Eleventh International Conference (20–25 Aug. 2018, Novosibirsk, Russia); Abstracts. Institute of Cytology and Genetics, Siberian Branch of Russian Academy of Sciences; Novosibirsk State University. – Novosibirsk: ICG SB RAS, 2018. - P. 200 / DOI 10.18699/BGRSSB-2018-170;

\item[\textbullet] \textbf{Bondareva O.}, Chetverikova R., Rayko M., Abramson N. I. /Molecular adaptations of subterranean rodents to underground lifestyle. //  Abstracts of the international conference on computer biology MCCMB`17 (July 27-30, 2017, Moscow) http://mccmb.belozersky.msu.ru/2017/index.html.

\end{enumerate}


\end{small}
%%\bibliography{biblio}
