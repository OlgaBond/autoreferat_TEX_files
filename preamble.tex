%%% Поля и разметка страницы %%%
\usepackage{lscape} % Для включения альбомных страниц
\usepackage{geometry} % Для последующего задания полей
\usepackage{float}

%%% Кодировки и шрифты %%%
\usepackage{cmap} % Улучшенный поиск русских слов в полученном pdf-файле
\usepackage[T2A]{fontenc} % Поддержка русских букв
\usepackage[utf8]{inputenc} % Кодировка utf8
\usepackage[english, russian]{babel} % Языки: русский, английский
%\usepackage{pscyr} % Нормальные шрифты

\usepackage{extsizes} % Возможность сделать 14-й шрифт
\usepackage{anyfontsize}

%%% Математические пакеты %%%
\usepackage{amsthm,amsfonts,amsmath,amssymb,amscd} % Математические дополнения от AMS
\usepackage{icomma} % "Умная" запятая: $0,2$ --- число, $0, 2$ --- перечисление

%%% Оформление абзацев %%%
\usepackage{indentfirst} % Красная строка

%%% Цвета %%%
\usepackage[usenames]{color}
\usepackage{color}
\usepackage{colortbl}

%%% Таблицы %%%
\usepackage{longtable} % Длинные таблицы
\usepackage{multirow,makecell,array} % Улучшенное форматирование таблиц

%%% Общее форматирование
%\usepackage[singlelinecheck=off,center]{caption} % Многострочные подписи
\usepackage{caption2}%форматирование подписей к плавающим объектам
\renewcommand{\captionlabeldelim}{.}% после названия объекта ставим точку.
\usepackage{soul} % Поддержка переносоустойчивых подчёркиваний и зачёркиваний

%%% Библиография %%%
%\usepackage{cite}

%%% Гиперссылки %%%
%\usepackage[plainpages=false,pdfpagelabels=false]{hyperref}
\usepackage[linktocpage=true,plainpages=false,pdfpagelabels=false]{hyperref}

%%% Изображения %%%
\usepackage{graphicx} % Подключаем пакет работы с графикой

%%% Опционально %%%
% следующий пакет может быть полезен, если надо ужать текст, чтобы сам текст не править, но чтобы места он занимал поменьше
\usepackage{savetrees}        

% этот пакет может быть полезен для печати текста брошюрой, сама с ним не разбиралась
%\usepackage[print]{booklet}

\newcommand{\R}{R} %Rlogo

\usepackage{wasysym}

%%%%%%%%%%%%%%%%%%%%%%%%%%%%%%%%%%%%%%%%%%%%%

%%% Макет страницы %%%


%\geometry{top=1.5cm,bottom=1.5cm,left=1.5cm,right=1cm}
\geometry{top=1.5cm,bottom=1.5cm,left=2cm,right=1.5cm}
%\oddsidemargin=-13pt
%\topmargin=-66pt
%\headheight=12pt
%\headsep=38pt
%\textheight=732pt
%\textwidth=484pt
%\marginparsep=14pt
%\marginparwidth=43pt
%\footskip=14pt
%\marginparpush=7pt
%\hoffset=0pt
%\voffset=0pt
%\paperwidth=597pt
%\paperheight=845pt
\parindent=1cm %размер табуляции (для красной строки) в начале каждого абзаца
\renewcommand{\baselinestretch}{1.25}
%\newfloat{scheme}{tb}{sch}

%%% Общая информация %%%
\author{Бондарева О.В.} % Фамилия И.О. автора

%%% Кодировки и шрифты %%%
%\renewcommand{\rmdefault}{ftm} % Включаем Times New Roman

%%% Выравнивание и переносы %%%
\sloppy
\clubpenalty=10000
\widowpenalty=10000


%%% Изображения %%%
\graphicspath{{images/}} % Пути к изображениям

%%% Цвета гиперссылок %%%
\definecolor{linkcolor}{rgb}{0,0,0}
\definecolor{citecolor}{rgb}{0,0,0}
\definecolor{urlcolor}{rgb}{0,0,0}
\hypersetup{
    colorlinks, linkcolor={linkcolor},
    citecolor={citecolor}, urlcolor={urlcolor}
}


%%% Библиография %%%
%\makeatletter
%\bibliographystyle{utf8gost705u} % Оформляем библиографию в соответствии с ГОСТ 7.0.5
%\renewcommand{\@biblabel}[1]{#1.} % Заменяем библиографию с квадратных скобок на точку:
%\makeatother

%% Библиография
%\usepackage[%
%backend=biber,% движок
%bibencoding=utf8,% кодировка bib файла
%%sorting=none,% настройка сортировки списка литературы
%%style=gost-authoryear,% стиль цитирования и библиографии (по ГОСТ)
%style=gost-authoryear,
%%maxitemnames=3,maxbibnames=12, %doesn't work
%sorting=lnyt,% настройка сортировки списка литературы % %doesn't work!
%maxnames=1,
%maxcitenames=2, %?
%minnames=1, %это число имён в цитате внутри текста
%uniquelist=false, %даже если первый автор один и тот же, сокращать до первого
%maxbibnames=10,
%movenames=false,ibidtracker=false, %koupriyanov
%language=auto,% получение языка из babel/polyglossia, default: autobib % если ставить autocite или auto, то цитаты в тексте с указанием страницы, получат указание страницы на языке оригинала
%autolang=other,% многоязычная библиография
%%clearlang=true,% внутренний сброс поля language, если он совпадает с языком из babel/polyglossia
%%defernumbers=true,% нумерация проставляется после двух компиляций, зато позволяет выцеплять библиографию по ключевым словам и нумеровать не из большего списка
%sortcites=false,
%%sortcites=true,% сортировать номера затекстовых ссылок при цитировании (если в квадратных скобках несколько ссылок, то отображаться будут отсортированно, а не абы как)
%doi=false,% Показывать или нет ссылки на DOI
%isbn=false,% Показывать или нет ISBN
%useauthor=true, %that's it! Если не ставить эту опцию, то во многих случаях будет использовано название вместо первого автора
%%babel=hyphen,  %no-no! You would have 'и др.' for english papers
%%babel=other,
%firstinits=true, 
%uniquename=false,
%]{biblatex}
%
%\usepackage{totcount}
%\usepackage{ifthen}
%\newcounter{bibliosel}
%\setcounter{bibliosel}{1}           % 0 --- встроенная реализация с загрузкой файла через движок bibtex8; 1 --- реализация пакетом biblatex через движок biber
%
%%%% Выбор реализации %%%
%\ifthenelse{\equal{\thebibliosel}{0}}{%
%	%%% Реализация библиографии встроенными средствами посредством движка bibtex8 %%%

%%% Пакеты %%%
\usepackage{cite}                                   % Красивые ссылки на литературу


%%% Стили %%%
\bibliographystyle{../BibTeX-Styles/utf8gost71u}    % Оформляем библиографию по ГОСТ 7.1 (ГОСТ Р 7.0.11-2011, 5.6.7)

\makeatletter
\renewcommand{\@biblabel}[1]{#1.}   % Заменяем библиографию с квадратных скобок на точку
\makeatother
%% Управление отступами между записями
%% требует etoolbox 
%% http://tex.stackexchange.com/a/105642
%\patchcmd\thebibliography
% {\labelsep}
% {\labelsep\itemsep=5pt\parsep=0pt\relax}
% {}
% {\typeout{Couldn't patch the command}}

%%% Цитирование %%%
\renewcommand\citepunct{;\penalty\citepunctpenalty%
    \hskip.13emplus.1emminus.1em\relax}                % Разделение ; при перечислении ссылок (ГОСТ Р 7.0.5-2008)


%%% Создание команд для вывода списка литературы %%%
\newcommand*{\insertbibliofull}{
\bibliography{../biblio/othercites,../biblio/authorpapersVAK,../biblio/authorpapers,../biblio/authorconferences}         % Подключаем BibTeX-базы % После запятых не должно быть лишних пробелов — он "думает", что это тоже имя пути
}

\newcommand*{\insertbiblioauthor}{
\bibliography{../biblio/authorpapersVAK,../biblio/authorpapers,../biblio/authorconferences}         % Подключаем BibTeX-базы % После запятых не должно быть лишних пробелов — он "думает", что это тоже имя пути
}

\newcommand*{\insertbiblioother}{
\bibliography{../biblio/othercites}         % Подключаем BibTeX-базы
}


%% Счётчик использованных ссылок на литературу, обрабатывающий с учётом неоднократных ссылок
%% Требуется дважды компилировать, поскольку ему нужно считать актуальный внешний файл со списком литературы
\newtotcounter{citenum}
\def\oldcite{}
\let\oldcite=\bibcite
\def\bibcite{\stepcounter{citenum}\oldcite}
  % Встроенная реализация с загрузкой файла через движок bibtex8
%}{
%	%%% Реализация библиографии пакетами biblatex и biblatex-gost с использованием движка biber %%%

\usepackage{csquotes} % biblatex рекомендует его подключать. Пакет для оформления сложных блоков цитирования.
	

%%% Загрузка пакета с основными настройками %%%
\usepackage[%
backend=biber,% движок
bibencoding=utf8,% кодировка bib файла
%sorting=none,% настройка сортировки списка литературы
%style=gost-authoryear,% стиль цитирования и библиографии (по ГОСТ)
style=gost-authoryear,
%maxitemnames=3,maxbibnames=12, %doesn't work
sorting=lnyt,% настройка сортировки списка литературы % %doesn't work!
maxnames=1,
maxcitenames=2, %?
minnames=1, %это число имён в цитате внутри текста
uniquelist=false, %даже если первый автор один и тот же, сокращать до первого
maxbibnames=10,
movenames=false,ibidtracker=false, %koupriyanov
language=auto,% получение языка из babel/polyglossia, default: autobib % если ставить autocite или auto, то цитаты в тексте с указанием страницы, получат указание страницы на языке оригинала
autolang=other,% многоязычная библиография
%clearlang=true,% внутренний сброс поля language, если он совпадает с языком из babel/polyglossia
%defernumbers=true,% нумерация проставляется после двух компиляций, зато позволяет выцеплять библиографию по ключевым словам и нумеровать не из большего списка
sortcites=false,
%sortcites=true,% сортировать номера затекстовых ссылок при цитировании (если в квадратных скобках несколько ссылок, то отображаться будут отсортированно, а не абы как)
doi=false,% Показывать или нет ссылки на DOI
isbn=false,% Показывать или нет ISBN
useauthor=true, %that's it! Если не ставить эту опцию, то во многих случаях будет использовано название вместо первого автора
%babel=hyphen,  %no-no! You would have 'и др.' for english papers
%babel=other,
firstinits=true, 
uniquename=false,
]{biblatex}


\nohyphenation

% %et al / и др. без квадратных скобок
%get rid of []

\renewbibmacro*{name:andothers}{%
  \ifboolexpr{
    test {\ifnumequal{\value{listcount}}{\value{liststop}}}
    and
    test \ifmorenames
  }
    {\ifnumgreater{\value{liststop}}{1}
       {\finalandcomma}
       {}%
     \andothersdelim\bibstring{andothers}}
     %here should be some code to stop 
    {}}

%\DefineBibliographyStrings{english}{%
%  andothers = \bibstring{andothers},
%}

\DefineBibliographyStrings{english}{%
    andothers = {\em et\addabbrvspace al\adddot}
}


%http://tex.stackexchange.com/a/141831/79756
%There is a way to automatically map the language field to the langid field. The following lines in the preamble should be enough to do that.
%This command will copy the language field into the langid field and will then delete the contents of the language field. The language field will only be deleted if it was successfully copied into the langid field.
\DeclareSourcemap{ %модификация bib файла перед тем, как им займётся biblatex 
    \maps{
        \map{% перекидываем значения полей language в поля langid, которыми пользуется biblatex
            \step[fieldsource=language, fieldset=langid, origfieldval, final]
            \step[fieldset=language, null]
        }
        \map{% перекидываем значения полей numpages в поля pagetotal, которыми пользуется biblatex
            \step[fieldsource=numpages, fieldset=pagetotal, origfieldval, final]
            \step[fieldset=pagestotal, null]
        }
        \map{% если в поле medium написано "Электронный ресурс", то устанавливаем поле media. которым пользуется biblatex в значение eresource
            \step[fieldsource=medium,
            match=\regexp{Электронный\s+ресурс},
            final]
            \step[fieldset=media, fieldvalue=eresource]
        }
        \map[overwrite]{% стираем значения всех полей issn
            \step[fieldset=issn, null]
        }
        \map[overwrite]{% стираем значения всех полей abstract, поскольку ими не пользуемся, а там бывают "неприятные" латеху символы
            \step[fieldsource=abstract]
            \step[fieldset=abstract,null]
        }
        \map[overwrite]{ % переделка формата записи даты
            \step[fieldsource=urldate,
            match=\regexp{([0-9]{2})\.([0-9]{2})\.([0-9]{4})},
            replace={$3-$2-$1$4}, % $4 вставлен исключительно ради нормальной работы программ подсветки синтаксиса, которые некорректно обрабатывают $ в таких конструкциях
            final]
        }
        \map[overwrite]{ % добавляем ключевые слова, чтобы различать источники
            \perdatasource{othercites.bib}
            \step[fieldset=keywords, fieldvalue={bibliofull}]
        }
        }
%        \map[overwrite]{% стираем значения всех полей series
%            \step[fieldset=series, null]
%        }
        \map[overwrite]{% перекидываем значения полей howpublished в поля organization для типа online
            \step[typesource=online, fieldsource=howpublished, fieldset=organization, origfieldval, final]
            \step[fieldset=howpublished, null]
        }
        % Так отключаем [Электронный ресурс]
%        \map[overwrite]{% стираем значения всех полей media=eresource
%            \step[fieldsource=media,
%            match={eresource},
%            final]
%            \step[fieldset=media, null]
%        }
    }



% %Нумерованный список при использовании authoryear

\defbibenvironment{bibliography}
  {\enumerate
     {}
     {\setlength{\leftmargin}{\bibhang}%
      \setlength{\itemindent}{-\leftmargin}%
      \setlength{\itemsep}{\bibitemsep}%
      \setlength{\parsep}{\bibparsep}}}
  {\endenumerate}
  {\item}


%%% Правка записей типа thesis, чтобы дважды не писался автор
%\DeclareBibliographyDriver{thesis}{%
%  \usebibmacro{bibindex}%
%  \usebibmacro{begentry}%
%  \usebibmacro{heading}%
%  \newunit
%  \usebibmacro{author}%
%  \setunit*{\labelnamepunct}%
%  \usebibmacro{thesistitle}%
%  \setunit{\respdelim}%
%  %\printnames[last-first:full]{author}%Вот эту строчку нужно убрать, чтобы автор диссертации не дублировался
%  \newunit\newblock
%  \printlist[semicolondelim]{specdata}%
%  \newunit
%  \usebibmacro{institution+location+date}%
%  \newunit\newblock
%  \usebibmacro{chapter+pages}%
%  \newunit
%  \printfield{pagetotal}%
%  \newunit\newblock
%  \usebibmacro{doi+eprint+url+note}%
%  \newunit\newblock
%  \usebibmacro{addendum+pubstate}%
%  \setunit{\bibpagerefpunct}\newblock
%  \usebibmacro{pageref}%
%  \newunit\newblock
%  \usebibmacro{related:init}%
%  \usebibmacro{related}%
%  \usebibmacro{finentry}}


%\newbibmacro{string+doi}[1]{% новая макрокоманда на простановку ссылки на doi
%    \iffieldundef{doi}{#1}{\href{http://dx.doi.org/\thefield{doi}}{#1}}}
%
%\renewcommand*{\mkgostheading}[1]{\usebibmacro{string+doi}{#1}} % ссылка на doi с авторов. стоящих впереди записи
%\renewcommand*{\mkgostheading}[1]{#1} % только лишь убираем курсив с авторов
%\DeclareFieldFormat{title}{\usebibmacro{string+doi}{#1}} % ссылка на doi с названия работы
%\DeclareFieldFormat{journaltitle}{\usebibmacro{string+doi}{#1}} % ссылка на doi с названия журнала
% Убрать тире из разделителей элементов в библиографии:
%\renewcommand*{\newblockpunct}{%
%    \addperiod\space\bibsentence}%block punct.,\bibsentence is for vol,etc.

%%% Возвращаем запись «Режим доступа» %%%
%\DefineBibliographyStrings{english}{%
%    urlfrom = {Mode of access}
%}
%\DeclareFieldFormat{url}{\bibstring{urlfrom}\addcolon\space\url{#1}}

%%% Set low penalties for breaks at uppercase letters and lowercase letters
%\setcounter{biburllcpenalty}{500} %управляет разрывами ссылок после маленьких букв RTFM biburllcpenalty
%\setcounter{biburlucpenalty}{3000} %управляет разрывами ссылок после больших букв, RTFM biburlucpenalty

%%% Список литературы с красной строки (без висячего отступа) %%%
%\defbibenvironment{bibliography} % переопределяем окружение библиографии из gost-numeric.bbx пакета biblatex-gost
%  {\list
%     {\printtext[labelnumberwidth]{%
%	\printfield{prefixnumber}%
%	\printfield{labelnumber}}}
%     {%
%      \setlength{\labelwidth}{\labelnumberwidth}%
%      \setlength{\leftmargin}{0pt}% default is \labelwidth
%      \setlength{\labelsep}{\widthof{\ }}% Управляет длиной отступа после точки % default is \biblabelsep
%      \setlength{\itemsep}{\bibitemsep}% Управление дополнительным вертикальным разрывом между записями. \bibitemsep по умолчанию соответствует \itemsep списков в документе.
%      \setlength{\itemindent}{\bibhang}% Пользуемся тем, что \bibhang по умолчанию принимает значение \parindent (абзацного отступа), который переназначен в styles.tex
%      \addtolength{\itemindent}{\labelwidth}% Сдвигаем правее на величину номера с точкой
%      \addtolength{\itemindent}{\labelsep}% Сдвигаем ещё правее на отступ после точки
%      \setlength{\parsep}{\bibparsep}%
%     }%
%      \renewcommand*{\makelabel}[1]{\hss##1}%
%  }
%  {\endlist}
%  {\item}

%%% Подключение файлов bib %%%
%\bibliography{othercites}
\addbibresource{othercites.bib}


%% Счётчик использованных ссылок на литературу, обрабатывающий с учётом неоднократных ссылок
%http://tex.stackexchange.com/a/66851/79756
%\newcounter{citenum}
\newtotcounter{citenum}
\makeatletter
\defbibenvironment{counter} %Env of bibliography
  {\setcounter{citenum}{0}%
  \renewcommand{\blx@driver}[1]{}%
  } %what is doing at the beginining of bibliography. In your case it's : a. Reset counter b. Say to print nothing when a entry is tested.
  {} %Здесь то, что будет выводиться командой \printbibliography. \thecitenum сюда писать не надо
  {\stepcounter{citenum}} %What is printing / executed at each entry.
\makeatother
\defbibheading{counter}{}



\newtotcounter{citeauthorvak}
\makeatletter
\defbibenvironment{countauthorvak} %Env of bibliography
{\setcounter{citeauthorvak}{0}%
    \renewcommand{\blx@driver}[1]{}%
} %what is doing at the beginining of bibliography. In your case it's : a. Reset counter b. Say to print nothing when a entry is tested.
{} %Здесь то, что будет выводиться командой \printbibliography. Обойдёмся без \theciteauthorvak в нашей реализации
{\stepcounter{citeauthorvak}} %What is printing / executed at each entry.
\makeatother
\defbibheading{countauthorvak}{}

\newtotcounter{citeauthornotvak}
\makeatletter
\defbibenvironment{countauthornotvak} %Env of bibliography
{\setcounter{citeauthornotvak}{0}%
    \renewcommand{\blx@driver}[1]{}%
} %what is doing at the beginining of bibliography. In your case it's : a. Reset counter b. Say to print nothing when a entry is tested.
{} %Здесь то, что будет выводиться командой \printbibliography. Обойдёмся без \theciteauthornotvak в нашей реализации
{\stepcounter{citeauthornotvak}} %What is printing / executed at each entry.
\makeatother
\defbibheading{countauthornotvak}{}

\newtotcounter{citeauthorconf}
\makeatletter
\defbibenvironment{countauthorconf} %Env of bibliography
{\setcounter{citeauthorconf}{0}%
    \renewcommand{\blx@driver}[1]{}%
} %what is doing at the beginining of bibliography. In your case it's : a. Reset counter b. Say to print nothing when a entry is tested.
{} %Здесь то, что будет выводиться командой \printbibliography. Обойдёмся без \theciteauthorconf в нашей реализации
{\stepcounter{citeauthorconf}} %What is printing / executed at each entry.
\makeatother
\defbibheading{countauthorconf}{}

\newtotcounter{citeauthor}
\makeatletter
\defbibenvironment{countauthor} %Env of bibliography
{\setcounter{citeauthor}{0}%
    \renewcommand{\blx@driver}[1]{}%
} %what is doing at the beginining of bibliography. In your case it's : a. Reset counter b. Say to print nothing when a entry is tested.
{} %Здесь то, что будет выводиться командой \printbibliography. Обойдёмся без \theciteauthor в нашей реализации
{\stepcounter{citeauthor}} %What is printing / executed at each entry.
\makeatother
\defbibheading{countauthor}{}





%%% Создание команд для вывода списка литературы %%%
\newcommand*{\insertbibliofull}{
\printbibliography[keyword=bibliofull,section=0]
\printbibliography[heading=counter,env=counter,keyword=bibliofull,section=0]
}


		% % %SORTING: Russian first!
		\DeclareSortingScheme{lnyt}{
		  \sort{
		    \field{presort}
		  }
		  \sort[final]{
		    \field{sortkey}
		  }
		  \sort[direction=descending]{\field{langid}}
		  \sort{
		    \field{sortname}
		    \field{author}
		    \field{editor}
		    \field{translator}
		    \field{sorttitle}
		    \field{title}
		  }
		  \sort{
		    \field{sortyear}
		    \field{year}
		  }
		  \sort{
		    \field{sorttitle}
		    \field{title}
		  }
		  \sort{
		    \field[padside=left,padwidth=4,padchar=0]{volume}
		    \literal{0000}
		  }
		}
		% % %

    % Реализация пакетом biblatex через движок bibers
%	
%}
